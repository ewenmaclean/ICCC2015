\section{Evaluation and Outlook}
\label{sec:eval}

In addition to motivating a further investigation of the role blending
can play in proofs, Galois theory, discussed above, is paradigmatic
for other reasons.

As Lautman writes in ``Ascent towards the Absolute'' \cite[p. 126]{lautman2011mathematics}:
\begin{quote}
What is characteristic of the movement of the theories that will be
considered here is the existence of an end conceived in advance as a
term of the ascent.
\end{quote}

This is reminiscent of our notion of coherence criteria that apply to
the blend.  For example, it seems useful to imagine that we would
progressively move from porcupine+lion to the perfected
\emph{porculione}.  Here, instead of field automorphisms that preserve
the structure of the base field, we would look at mappings that
preserve other properties that exist in the underlying domain.
Porculiones would have four feet, would be mammals, and would
presumably be omnivores with a preference for the taste of well-cured
game.

\cite[pp. 227--228]{deleuze1994difference} follows
Lautman in enthusiastically endorsing the Galoisian approach to
mathematics:
\begin{quote}
[T]he fact that an equation cannot be solved algebraically, for
example, is no longer discovered as a result of empirical research or
by trial and error, but as a result of the characteristics of the
groups and partial resolvents which constitute the synthesis of the
problem and its conditions (an equation is solveable only by algebraic
means -- in other words, by radicals, when the partial resolvents are
binomial equations and the indices of the groups are prime numbers).
The theory of problems is completely transformed and at last grounded,
since we are no longer in the classic master-pupil situation where the
pupil understands and follows a problem only to the extent that the
master already knows the solution and provides the necessary
adjunctions.  For, as Georges Verriest remarks, the group of an
equation does not characterise at a given moment what we know about
its roots, but the objectivity of what we do not know about them.
Conversely, this non-knowledge is no longer a negative or an
insufficiency but a rule or \emph{something to be learnt} which
corresponds to a fundamental dimension of the object.
\end{quote}

Although there is a commonality between blending and the
Galoisian approach insofar as progressive refinement carries us toward a
``perfected'' conclusion, Deleuze's enthusiasm about the pedagogical
situation would be significantly cooled here.  It would seem, in many
of our examples, that we only make progress ``to the extent that the
master already knows the solution and provides the necessary
adjunctions.''

However, this apparent infelicity may be less of a thick obstacle than
it would initially appear.  What seems to be most needed in the system
is a notion of \emph{question}, which would recover Lautman's basic
thrust: ``Scientific or not, every question has built in some
assumptions about the form of the answer'' \cite{larvor2011albert}.
Furthermore, an experimental approach would allow us to embed
evaluation in the system itself.  

In this connection, the ideas discussed above about blends as steps in
a proof would provide a useful training ground for further
development.  It would certainly be possible to conceptualise
\emph{modus ponens} or \emph{modus tollens}, etc., as rules for
creating blends, but the ability to detect which items to blend is
also critically important.  \cite{AbrSad14} develops a category
theoretical approach to linguistic anaphora that could be useful for
the purpose of fitting different mathematical ``components'' together
in a sensible manner.  \cite{DBLP:journals/corr/GanesalingamG13} is a
recently developed system that uses a Logic of Computable Functions
approach to prove simple theorems.  A system that could selectively
experiment with the rules it uses and learn to associate different
techniques with different types of problems would be a useful
synthesis of these ideas.

%%% Local Variables: 
%%% mode: latex
%%% TeX-master: "mathsICCC"
%%% End: 

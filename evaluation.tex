\section{Evaluation and Outlook}
\label{sec:eval}

\subsection{Review of the current offering}

\begin{itemize}
\item[(a)] We began the paper with the reconstruction of certain
mathematical objects, showing the technical feasibility of the
approach. 
\item[(b)] The more advanced example at the centre of the paper
  illustrates how this sort of reconstruction relates to mathematical
  practice.
\item[(c)] A future-oriented example exposes some technical
  challenges, while suggesting that blending could offer a novel
  approach to computer mathematics.
\end{itemize}

\subsection{Broader issues in evaluation}

In addition to motivating a further investigation of the role blending
can play in proofs, Galois theory, discussed above, is paradigmatic
for other reasons.  This discussion draws on the early 20th Century
writings of Albert Lautman on the philosophy of mathematics and the
subsequent interpretation of this work by Gilles Deleuze.  It uses
these ideas to propose an approach to embedding evaluation within the
system itself.

Concerning the common features of Galois theory, class field theory,
and the development of the universal covering surface in Riemann
geometry, \cite[p. 126]{lautman2011mathematics} writes:
\begin{quote}
What is characteristic of the movement of the theories that will be
considered here is the existence of an end conceived in advance as a
term of the ascent.
\end{quote}

% This is reminiscent of our notion of coherence criteria that apply to
This is reminiscent of our notion of internal evaluation that apply to
the blend.  To illustrate, let us briefly imagine how we would use
blending techniques to move from porcupine+lion to the perfected
\emph{porculione}.  Here, instead of field automorphisms that preserve
mathematical structure and fix certain designated elements, we would
look for mappings that preserve other properties that exist in the
underlying domain.  Porculiones would presumably have four feet, would
be mammals, and would be omnivores; they should also be viable living
creatures.

\cite[pp. 227--228]{deleuze1994difference} follows
Lautman in enthusiastically endorsing the Galoisian approach to
mathematics:
\begin{quote}
[T]he fact that an equation cannot be solved algebraically, for
example, is no longer discovered as a result of empirical research or
by trial and error, but as a result of the characteristics of the
groups and partial resolvents which constitute the synthesis of the
problem and its conditions (an equation is solveable only by algebraic
means -- in other words, by radicals, when the partial resolvents are
binomial equations and the indices of the groups are prime numbers).
The theory of problems is completely transformed and at last grounded,
since we are no longer in the classic master-pupil situation where the
pupil understands and follows a problem only to the extent that the
master already knows the solution and provides the necessary
adjunctions.  For, as Georges Verriest remarks, the group of an
equation does not characterise at a given moment what we know about
its roots, but the objectivity of what we do not know about them.
Conversely, this non-knowledge is no longer a negative or an
insufficiency but a rule or \emph{something to be learnt} which
corresponds to a fundamental dimension of the object.
\end{quote}

Although there is a commonality between blending and the
Galoisian approach insofar as progressive refinement carries us toward a
``perfected'' conclusion, Deleuze's enthusiasm about the pedagogical
situation would be significantly cooled here.  It would seem, in many
of our examples, that we only make progress ``to the extent that the
master already knows the solution and provides the necessary
adjunctions.''

However, this apparent infelicity may be less of a thick obstacle than
it would initially appear.  What seems to be most needed is a notion
of a \emph{question} inside the system.  This would recover Lautman's
basic thrust: ``Scientific or not, every question has built in some
assumptions about the form of the answer'' \cite{larvor2011albert}.
In short, an experimental approach in which the system \emph{asks} and
\emph{answers} questions would embed key aspects for evaluation in the
system itself.

\subsection{Future work}

The idea of using blending to carry out steps in a proof would provide
a useful training ground for further development.  The primary problem
is: If blending is the realisation of ``combinatorial creativity'' how
will we avoid being swamped by the combinatorial explosion of possible
things to combine?  The first challenge is thus fitting different
mathematical components together in a sensible manner.  A related
challenge would apply when modifying the system to selectively
experiment with the rules it uses.  The objective in this case would
be for the system to learn to associate different (useful) techniques
with different types of problems.
%% Some reference to the music and other technology for blending here?

%%% Local Variables: 
%%% mode: latex
%%% TeX-master: "mathsICCC"
%%% End: 

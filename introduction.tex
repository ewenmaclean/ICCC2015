\section{Introduction}
\label{sec:intro}

We are concerned with creativity in mathematics: creativity
as evinced by human and artificial mathematicians,
individually and collectively.

Work on \emph{conceptual blending} has been much influenced
by \textcite{Fau98}.
More recently, the centrality of conceptual blending to creativity
has been stressed by \textcite{MTurner14}, where he writes:
\begin{quote}
  \dots the human spark comes from our advanced ability to \emph{blend} ideas
  to make \emph{new} ideas. Blending is the origin of ideas.%
  \hfill \parencite[p 2]{MTurner14}
\end{quote}
The claim is that blending in this sense is a general human cognitive
ability, and as such applies to mathematics, as much as to art, poetry,
music and so on (see for example \textcite{MTurner05}).

The place of mathematics and the sciences among creative endeavours
has been stressed by the literary critic George Steiner:
\begin{quote}
  It is in mathematics and the sciences that the concepts of
creation and of invention, of intuition and of discovery,
exhibit the most immediate, visible force.
\flushright{\citet[p~145]{Ste01}}
\end{quote}
% A recent survey of examples of blending in mathematics by
% \textcite{Al11i}.  
Blending involves recognising features common
to mathematical concepts, even when expressed in different
terminology.  The role of mathematical analogy in creative mathematics
is well expressed by \textcite{Weil60}, and a general plea for
analogical reasoning within science in \cite{ArbibHesse86}.

We are investigating \emph{computational} accounts of mathematical
creativity, taking conceptual blending as a key ingredient.  The work
%of Goguen \parencite{Gog99,Gog05b,Gog05} 
of \textcite{Gog99,Gog05b} 
has provided a general framework for
comparison of conceptual spaces, and computation of blends.  This
enables the use of richer representation formalisms, and so closer to
contemporary mathematics than previous computational realisations of
blending, such as in \textcite{Pereira2007}.

This paper deals with the creative process in mathematics, as modelled
along the lines above. We focus on the use of blending within a single
process, searching for blends satisfying some evaluation criteria,
from a starting point of some given conceptual spaces.
% next omitted from camera ready version --
% reviewers request more on the cognitive side, need the space!
% --
% The issues
% involved in social interaction between the agents concerned in more
% realistic scenarios are for consideration elsewhere, but are briefly
% addressed below.

While cognitive issues are important to us, this paper
is focused on issues in representation and representation change;
there are however brief comments on cognition in the conclusions.

We start by providing some background, followed by an example to
illustrate the components involved in our approach. A historical
example based on Georg Cantor's work follows.  The most extended example was
carried out by a pure mathematician (D.\ G{\'o}mez-Ram{\'{\i}}rez), working in a
domain close to his own; in this case, the blend mechanism threw up
some unexpected properties, which provoked new work by the
mathematician.

Subsequently we give some more speculative thoughts on where this work
can go in the future, by considering Galois theory as a test-bed.
Finally, we discuss the evaluation of work along these lines, and
give some conclusions.

%%% Local Variables: 
%%% mode: latex
%%% TeX-master: "mathsICCC"
%%% End: 

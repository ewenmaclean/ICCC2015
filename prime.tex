\section{Prime Ideals as a blend}
\label{sec:prime_ideals}

\subsection*{Introduction}
One of the most fundamental concepts of modern mathematics, which is
the basis of commutative algebra and a seminal ingredient of the
language of schemes in modern algebraic geometry, is that of
\emph{prime ideal} \parencite{EGAI,eisenbud}.

% Our approach to
% blending is the one adopted by Goguen in terms of colimits
% \parencite{Gog99,Goguen01,Gog05c}.
The terminology ``prime ideal'' relates to the older notion of ``prime
number''. The initial aim of this work was to look for a blend between
prime numbers (from the integers) and the ideals of a commutative
ring, to see what would emerge. It turned out that \emph{the blend process},
along with providing a definition for prime ideals, also \emph{suggested 
an unexpected concept in the context of rings, namely the notion of
Containment Division Ring (CDR)}. In turn, this prompted questions and
proofs about this concept -- thus running the blend (space prevents
description of this step below).

We present a first blend involving weakening, followed by a second
blend from fuller input spaces, where the emergent concept of CDR
appears.

% In this section, we will recover the concept of prime ideal of a
% commutative ring with unity as a sort of partial (or weakened) colimit
% (i.e.\ a colimit just taking into consideration some axioms of the
% input theories) between the concepts of an ideal of a commutative ring
% with unity (enriched with the collection of all the ideals of the
% corresponding ring) and the concept of a prime number of the integers.

% In fact, in order to obtain the desired space it is enough to consider
% a more general version of the prime numbers (in our case a partial
% version), namely, a monoid $(Z,*,1)$ with an ``special'' divisibility
% relation $\dannydiv$. Besides, the generic space would capture just
% the syntactic correspondences that we wish to identity in the blending
% space, since the blend would be basically the union of the collection
% of axioms given on each space, doing the corresponding
% identifications.

% Besides, by doing just slight modifications to the input conceptual
% spaces, i.e., by adding explicitly the upside-down divisibility
% condition to the second conceptual space we obtain as blend the space
% of prime ideals (and ideals) over a commutative ring satisfying
% additionally a new condition called the containment-division
% condition. We called this class of rings CDR (containment-division
% rings).

% Finally, we present the resulting axiomatization of an implementation
% done in HETS \parencite{Mossakowskihets}, which gives the conceptual space
% prime ideals (and ideals) over CDR-s as a blend.

% We present the conceptual spaces from the standard "pure" mathematical
% point of view doing concurrently the corresponding translation into
% the setting of the Common Algebraic Specification Language
% (CASL) \parencite{BidoitMosses2004}.
\subsection{The first conceptual space}
Let $(R,+.*,0,1)$ be a commutative ring with unity (see the formal
definition and examples in \textcite{eisenbud}). Now, $R$ can be understood
as the sort containing the elements of the corresponding commutative
ring with unity.  An ideal $I$ is a subset of $R$ satisfying the
following axiom:
\[(\forall i,j\in I)(\forall r\in R)(i+(-j)\in I \wedge r*i\in I).\]

Let us define a unary relation (predicate) $isideal$
on the set (sort) of subsets of $P(R)$ corresponding to this definition.
% , as follows: $isideal(I)$
% if and only if $I$ is an ideal of $R$.
Now, we define
\[{\rm Id}(R)=\left\{A \in P(R) : isideal(A) \right\}.\]
% Here, ${\rm Spec}_IR$ is considered as a subsort of the sort $P(R)$.

Ideals are ``multiplied'' together using the following definition:
% There is one natural operation on ${\rm Id}(R)$, let us say
% $\cdot_{\iota}$, inherited in a natural way from the corresponding
% operations $+$ and $\cdot$ on $S$:
%
% Let $I,J\in {\rm Id}(R)$, then we define 
%

\begin{adjustbox}{width=\columnwidth}
  \(
  \displaystyle
  I\cdot_{\iota} J=\left\{\sum_{k=1}^ni_k\cdot j_k:n \in \mathbb{N}
  \wedge i_1, \ldots, i_n\in I \wedge j_1, \ldots, j_n\in J \right\}.
  \)
\end{adjustbox}
%\[I\cdot_{\iota} J=\left\{\sum_{k=1}^ni_k\cdot j_k:n \in \mathbb{N} \wedge i_1, \ldots, i_n\in I \wedge j_1, \ldots, j_n\in J \right\}.\]
%\[I\cdot_{\iota} J=\left\{\sum_{k=1}^ni_k\cdot j_k:n \in \mathbb{N} \wedge i_k\in I \wedge j_k\in J \right\}.\]
% \[I\cdot_{\iota} J=\left\{i_1\cdot j_1+...+i_n\cdot j_n:n \in \mathbb{N} \wedge i_k\in I \wedge j_k\in J \right\}.\]

% With this operation ${\rm Id}(R)$ forms a commutative monoid (i.e.\ we
% have commutativity, associativity and there exists a neutral element
% (in this case the ring)). However, this fact is irrelevant in our case
% for the blending process. As a matter of fact, 
The key property that
we want to keep in the blend is the one saying that this operation
has a neutral element $1_{\iota}$, which can be seen as an additional
notation for the ring, but with respect to this operation $\cdot_{\iota}$.
% instead of being the sort of elements of the ring, i.e., $R$.
%
On the other hand, we want to see the containment relation $\subseteq$
as a binary relation over the sort ${\rm Id}(R)$.

Summarizing, our first conceptual space consists of sorts $R,
\textrm{Id}(R)$ and $P(R)$; operations $+, *, 0_R, 1_R, 1_{\iota}$ and
$ \cdot_{\iota}$; and the relations $\subseteq$ and $isideal$.


% Here we add all the corresponding axioms defining $R$ as a commutative ring, the explicit former definition of $isideal$, ${\rm Id}(R)$ and $\cdot_{\iota}$; and the axiom guaranteeing that  $1_{\iota}$ is the neutral element for $\cdot_{\iota}$.

% It is important to note that although $R$ and $1_{\iota}$ are the same
% from the set-theoretical point of view, we choose two symbols in order
% to stress the particular function that the same set fulfils, i.e., as
% a collection of elements ($R$) or as a neutral element itself
% ($1_{\iota}$. Besides, it is more useful to do that for the purpose of
% an implementation (see Section \ref{implementation}). However, this
% distinction is irrelevant for our present pure-theoretical
% considerations.
 
Let us denote this space by $\mathbb{I}$.

\subsection{The second conceptual space}
Let $\mathbb{Z}$ be the set of the integer numbers. Here, we can
choose any axiomatization of them, since for the (partial) blend we
just take into account only the fact that $(\mathbb{Z},*,1)$ is a
commutative monoid.
% Or even simpler, we only use the fact that $1$ is
% the neutral element with the operation $*$. 
We define also an upside-down divisibility relation $\dannydiv$ defined as
%
%\[e\dannydiv g:=g|e.\]
$e\dannydiv g:=g|e$,  i.e.\ there exists an integer $c$ such that $e=cg$.
% (This makes the commonality more obvious notationally, though
% it is not necessary for the blending machinery.)
% We re-write the classical divisibility relation
% on this way in order to obtain the right primality condition on the
%blend.  
Let us define a unary relation $isprime$ on $\mathbb{Z}$ as
follows: for all $p\in \mathbb{Z}$, $isprime(p)$ holds if $p\neq 1$
% and the following (primality) condition holds:
and:
%
\[(\forall a,b\in \mathbb{Z})\: \bigl( (ab \dannydiv p)\rightarrow (a \dannydiv p \vee b \dannydiv p) \bigr).\] 
%\[(\forall a,b\in \mathbb{Z})\: ( (ab \dannydiv p)\rightarrow (a \dannydiv p \vee b \dannydiv p) ).\] 
Besides, we define the set (sort) of the prime numbers as 
\[ Prime=\left\{ p\in \mathbb{Z}: isprime(p)\right\}\]
% Now, it is an
% elementary fact to see that this condition is an equivalent form of
% the standard definition of prime number given in the classical number
% theory books as the numbers $p$ with set of divisors
% $\left\{1,-1,p,-p\right\}$ (see for example
% \cite{Apostol76}). 
In the CASL language, we consider
$\mathbb{Z}$ as the sort of the integer numbers, $*$ as a binary
operation, $prime$ as a predicate and $\dannydiv$ as a binary
relation, any of them defined over the sort $\mathbb{Z}$.
We denote this conceptual space by $\mathbb{P}$.


\subsection{The Generic Space}

The generic space $\mathbb{G}$ consists of a set (sort) $G$ with a
binary operation $*_G$, a neutral element $S$ and a binary relation
$\leq_G$.

%  Let us denote this space by $\mathbb{G}$.

\subsection{The Blending Morphisms}

The morphism to $\mathbb{I}$  uses:\\
$\varphi(G)={\rm Id}(R),\varphi(*_G)=*_{\iota},
\varphi(S)=1_{\iota}$ and $\varphi(\leq_G)=\subseteq$;\\
the morphism to $\mathbb{G}$ uses:\\ $\delta(G)=\mathbb{Z}, \delta(*_G)=*,
\delta(S)=1$ and $\delta(\leq_G)= \dannydiv$.

% Now, let us define the morphisms from the generic space into the two
% corresponding conceptual spaces. Let $\varphi: \mathbb{G}\rightarrow
% \mathbb{I}$ be the morphism induced by the following syntactic
% correspondences $\varphi(G)={\rm Id}(R),\varphi(*_G)=*_{\iota},
% \varphi(S)=1_{\iota}$ and $\varphi(\leq_G)=\subseteq$.
% \newline\indent Furthermore, let
% $\delta:=\mathbb{G}\rightarrow\mathbb{P}$ be the morphism induced by
% the syntactic correspondences $\delta(G)=\mathbb{Z}, \delta(*_G)=*,
% \delta(S)=1$ and $\delta(\leq_G)= \dannydiv$.  

\subsection{The Axiomatization of the Blending}

A straightforward colimit construction based on the input and generic
spaces above yields a consistent space with properties inherited both
from the prime elements into the integers and from the ideals of
commutative rings; one of the concepts is a notion of prime ideals,
another is that of CDR.%
\footnote{A ring $R$ is a Containment Division Ring (CDR) if for all
  ideals $I$ and $J$ of $R$, $I \subseteq J$ if and only if $J$
  divides $I$ (i.e.\ there exists an ideal $U$ such that
  $I=U\cdot_{\iota}J$).}  
Here we describe briefly a weakening of the
given spaces that makes the resultant blend more generally applicable.

% In the everyday research of the working mathematician it happens
% frequently that one starts to develop general theoretical frameworks
% by combining just some aspects of two particular theories but without
% considering the whole theories.
%
% For example, the development of
% differential geometry was obtained combining just some aspect of
% general and algebraic topology and some aspects of real analysis
% \cite{VelCad05}. The same happens with the methods use in
% analytic number theory which are a fusion of some components of
% elementary number theory and some of the real analysis techniques
% \cite{Apostol76}.
%
% Therefore, it is sometimes more natural in the daily mathematical research to
% obtain new concepts as ``partial'' combinations of two former ones,
% i.e., as combinations (blends) of just some axioms of the
% corresponding two theories.
 
% Thus, in our case, a partial blend will give us the desired
% concept. For example,
From the properties defining the integers we
transfer into the blend only the fact that $\mathbb{Z}$ is a set with
a binary operation $*$ having $1$ as neutral element and
$\dannydiv$ as a binary relation, without taking into account its formal definition.

% So, after using the same symbols for denoting the ring as a sort of
% elements or as the neutral element for the product of ideals
% $\cdot_G$, the blend has the form

% \[(S, +, *, 0_S, 1_S, G={\rm Id}(S), isprime,\]
% \[ Prime, \cdot_G, S=1_G, \subseteq)\] with all the corresponding
% axioms of the first conceptual space plus the translated version of
% the axiom defining the primality predicate after doing the
% corresponding symbolic identifications i.e., an element $P \in G$
Now after computing the colimit, we obtain that
any element $P \in G$
(i.e., an ideal of $S$) satisfies the predicate $isprime$ if and only
if
\begin{align*}
P\neq S \wedge (\forall X,Y\in G={\rm Id}(S)) & (X\cdot_{\iota}Y\subseteq P\\
                             & \rightarrow (X\subseteq P \vee Y\subseteq P)).
\end{align*}

% Now, it is an elementary exercise to see that this definition is
% equivalent to the fact that $P$ is a prime ideal of $S$, i.e.\ to the
% condition
% %
% \[ P\neq S \wedge (\forall a,b\in S)(ab\in B\rightarrow (a\in P \vee
% b\in P)).\]
Thus, the predicate $isprime$ turns out to be the
predicate characterizing the primality of ideals of $S$ and the set
(sort) $Prime$ turns out to be the set of prime ideals of $S$.

% Besides, we just consider the fact that the upside-down divisibility
% relation is a binary relation without taking into account the formal
% definition into the blend.

Using the weakened input spaces, the blending space consists of the
axioms assuring that $S$ is a commutative ring with unity, $G$ is the
set of ideals of $S$, $isprime$ is the predicate specifying primality
for ideals of $S$ and $Prime$ is the collection of all prime ideals of
$S$.


\subsection{Implementation for prime ideals over CDR-s as a blend}\label{implementation}

% It is a well-known fact that the principal ideal domains, or even the
% Dedekind domains are CDR-s (see for example \cite[Fundamental Theorem
% of AOK-s]{weissalgebraicnumber}).

% On the other hand, if $R$ is not a Dedekind domain, but for example a
% unique factorization domain (UFD), then $R$ is not, in general, a
% CDR. For example, when $R=\mathbb{Z}[T]$, one can check that the
% ideals $X=(2)$ and $Y=(2,T)$ gives a counterexample.

% This suggests that in the setting of commutative rings with unity the
% class of CDR-s could be an intermediate new class of rings.

% In this section we reconstruct the conceptual space of prime ideals of
    In this section we construct the concept of prime ideal over a CDR as a    
    blend of the conceptual space of ideals of a commutative ring with         
    unity and the conceptual space of the former second conceptual space where
    the axiom defining the upside-down divisibility          
    relation is restored.


It is worth mentioning again that the definition of CDR-s was obtained after
doing this implementation and therefore it could be seen as a form of
``creative'' result coming from the blending process.

% Here, it is important to point out that in this implementation we
% looked for a minimal set of axioms such that, at the same time, the
% semantic interpretation can be uniquely determined. It is always
% possible to construct an implementation with additional axioms given
% by properties that could be logically derived from the main axioms,
% (e.g. the set theoretical properties of the containment relation for
% subsets of a set) but these properties are secondary ones, meanwhile,
% the ones defining the arithmetic of the ring, of an ideal and of the
% set of ideals of the ring are the essential ones.

After computing the corresponding colimit in HETS and interpreting
"RingElt" as the sort containing the elements of the ring $S$, the
theory defining the blend corresponds to the axioms defining a CDR
(S), the set of all its ideals (Generic), the set all its prime ideals
(SimplePrime) and a primality predicate (IsPrime).  We present 
in Listing~\ref{listing:prime_colim}
just the theory corresponding to the colimit (with some details
omitted). 

% \begin{figure}[ht!]
% 
  \begin{listing}[!ht]
    \begin{mdframed}
     \begin{footnotesize}
      \begin{hetcasl}
% \KW{library} \Id{ideal\Ax{\_}colim}\\
% \\
% \KW{logic} \SId{CASL}\Id{\Ax{.}}\SId{SulFOL=}\\
% \\
\SPEC \=\SIdIndex{Spec} \Ax{=}\\
\> \SORTS \=\Id{Generic}, \Id{RingElt}, \Id{SimplePrime}, \Id{SubSetOfRing}\\
\> \SORTS \=\Id{SimplePrime} \Ax{<} \Id{Generic},Id{Generic} \Ax{<} \Id{SubSetOfRing}\\
\> \OPS \=\Ax{0,1,S} \Ax{:} \Id{RingElt}\\
% \> \OP \=\Ax{1} \Ax{:} \Id{RingElt}\\
% \> \OP \=\Id{S} \Ax{:} \Id{Generic}\\
\> \OP \=\Ax{\_\_}\Ax{*}\Ax{\_\_} \Ax{:} \=\Id{RingElt} \Ax{\times} \Id{RingElt} \Ax{\rightarrow} \Id{RingElt}\\
\> \OP \=\Ax{\_\_}\Ax{+}\Ax{\_\_} \Ax{:} \=\Id{RingElt} \Ax{\times} \Id{RingElt} \Ax{\rightarrow} \Id{RingElt}\\
\> \OP \=\Ax{\_\_}\Id{x}\Ax{\_\_} \Ax{:} \=\Id{Generic} \Ax{\times} \Id{Generic} \Ax{\rightarrow} \Id{Generic}\\
\> \PRED \=\Id{IsIdeal} \Ax{:} \Id{SubSetOfRing}\\
\> \PRED \=\Id{IsPrime} \Ax{:} \Id{Generic}\\
\> \PRED \=\Ax{\_\_}\Id{isIn}\Ax{\_\_} \Ax{:} \=\Id{RingElt} \Ax{\times} \Id{SubSetOfRing}\\
\> \PRED \=\Id{gcont} \Ax{:} \=\Id{Generic} \Ax{\times} \Id{Generic}\\
\> \PRED \=\Ax{\_\_}\Id{generates}\Ax{\_\_} \Ax{:} \=\Id{RingElt} \Ax{\times} \Id{Generic}\\
\> \Ax{\forall} \=\Id{I} \Ax{:} \Id{SubSetOfRing} \Ax{\bullet} \=\Id{I} \Ax{\in} \Id{Generic} \Ax{\Leftrightarrow} \Id{IsIdeal}(\Id{I})\\
\> \Ax{\forall} \=\Id{x} \Ax{:} \Id{Generic} \Ax{\bullet} \=\Id{x} \Id{x} \Id{S} \Ax{=} \Id{x}\\
\> \Ax{\forall} \=\Id{x} \Ax{:} \Id{Generic} \Ax{\bullet} \=\Id{S} \Id{x} \Id{x} \Ax{=} \Id{x}\\
\> \Ax{\forall} \=\Id{A}, \Id{B} \Ax{:} \Id{Generic} \\
\> \Ax{\bullet} \=\Id{gcont}(\=\Id{A}, \Id{B}) \Ax{\Leftrightarrow} \=\Ax{\forall} \Id{a} \Ax{:} \Id{RingElt} \Ax{\bullet} \=\Id{a} \Id{isIn} \Id{A} \Ax{\Rightarrow} \=\Id{a} \Id{isIn} \Id{B}\\
\> \Ax{\forall} \=\Id{x}, \Id{y} \Ax{:} \Id{RingElt} \Ax{\bullet} \=\Id{x} \Ax{+} \Id{y} \Ax{=} \=\Id{y} \Ax{+} \Id{x}\\
\> \KW{\%\%} \emph{and further ring axioms} \dots \\
% \> \Ax{\forall} \=\Id{x}, \Id{y}, \Id{z} \Ax{:} \Id{RingElt} \Ax{\bullet} \=(\=\Id{x} \Ax{+} \Id{y}) \Ax{+} \Id{z} \Ax{=} \=\Id{x} \Ax{+} (\=\Id{y} \Ax{+} \Id{z})\\
% \> \Ax{\forall} \=\Id{x} \Ax{:} \Id{RingElt} \Ax{\bullet} \=\Id{x} \Ax{+} \Ax{0} \Ax{=} \Id{x} \Ax{\wedge} \=\Ax{0} \Ax{+} \Id{x} \Ax{=} \Id{x}\\
% \> \Ax{\forall} \=\Id{x} \Ax{:} \Id{RingElt} \Ax{\bullet} \=\Ax{\exists} \Id{x'} \Ax{:} \Id{RingElt} \Ax{\bullet} \=\Id{x'} \Ax{+} \Id{x} \Ax{=} \Ax{0}\\
% \> \Ax{\forall} \=\Id{x}, \Id{y} \Ax{:} \Id{RingElt} \Ax{\bullet} \=\Id{x} \Ax{*} \Id{y} \Ax{=} \=\Id{y} \Ax{*} \Id{x}\\
% \> \Ax{\forall} \=\Id{x}, \Id{y}, \Id{z} \Ax{:} \Id{RingElt} \Ax{\bullet} \=(\=\Id{x} \Ax{*} \Id{y}) \Ax{*} \Id{z} \Ax{=} \=\Id{x} \Ax{*} (\=\Id{y} \Ax{*} \Id{z})\\
% \> \Ax{\forall} \=\Id{x} \Ax{:} \Id{RingElt} \Ax{\bullet} \=\Id{x} \Ax{*} \Ax{1} \Ax{=} \Id{x} \Ax{\wedge} \=\Ax{1} \Ax{*} \Id{x} \Ax{=} \Id{x}\\
% \> \Ax{\forall} \=\Id{x}, \Id{y}, \Id{z} \Ax{:} \Id{RingElt} \Ax{\bullet} \=(\=\Id{x} \Ax{+} \Id{y}) \Ax{*} \Id{z} \Ax{=} \=(\=\Id{x} \Ax{*} \Id{z}) \Ax{+} (\=\Id{y} \Ax{*} \Id{z})\\
% \> \Ax{\forall} \=\Id{x}, \Id{y}, \Id{z} \Ax{:} \Id{RingElt} \Ax{\bullet} \=\Id{z} \Ax{*} (\=\Id{x} \Ax{+} \Id{y}) \Ax{=} \=(\=\Id{z} \Ax{*} \Id{x}) \Ax{+} (\=\Id{z} \Ax{*} \Id{y}) \\
\> \Ax{\forall} \=\Id{I} \Ax{:} \Id{SubSetOfRing} \\
\> \Ax{\bullet} \=\Id{IsIdeal}(\Id{I}) \\
\>\> \Ax{\Leftrightarrow} \=\Ax{\forall} \Id{a}, \Id{b}, \Id{c} \Ax{:} \Id{RingElt} \\
\>\>\> \Ax{\bullet} \=(\=(\=\Id{a} \Id{isIn} \Id{I} \Ax{\Rightarrow} \=\Id{a} \Id{isIn} \Id{S}) \Ax{\wedge} \=\Ax{0} \Id{isIn} \Id{I}) \\
\>\>\>\> \Ax{\wedge} (\=\Id{a} \Id{isIn} \Id{I} \Ax{\wedge} \=\Id{c} \Id{isIn} \Id{S} \Ax{\Rightarrow} \=\Id{c} \Ax{*} \Id{a} \Id{isIn} \Id{I}) \\
\>\>\>\> \Ax{\wedge} (\=\Id{a} \Id{isIn} \Id{I} \Ax{\wedge} \=\Id{b} \Id{isIn} \Id{I} \Ax{\wedge} \=\Id{c} \Id{isIn} \Id{S} \Ax{\wedge} \=\Id{b} \Ax{+} \Id{c} \Ax{=} \Ax{0} \\
\>\>\>\>\> \Ax{\Rightarrow} \=\Id{a} \Ax{+} \Id{c} \Id{isIn} \Id{I})\\
\> \Ax{\forall} \Id{a} \Ax{:} \Id{RingElt}; \=\Id{A} \Ax{:} \Id{Generic} \\
\> \KW{\%\%} \emph{and axioms for \Ax{generates} and \Ax{x}} \dots \\
% \> \Ax{\bullet} \=\Id{a} \Id{generates} \Id{A} \\
% \>\> \Ax{\Leftrightarrow} \=\Ax{\forall} \Id{c} \Ax{:} \Id{RingElt} \Ax{\bullet} \=\Id{c} \Id{isIn} \Id{A} \Ax{\Rightarrow} \=\Ax{\exists} \Id{d} \Ax{:} \Id{RingElt} \Ax{\bullet} \=\Id{c} \Ax{=} \=\Id{a} \Ax{*} \Id{d}\\
% \> \Ax{\forall} \Id{A}, \Id{B} \Ax{:} \Id{Generic}; \=\Id{a}, \Id{b} \Ax{:} \Id{RingElt} \\
% \> \Ax{\bullet} \=\Id{a} \Id{isIn} \Id{A} \Ax{\wedge} \=\Id{b} \Id{isIn} \Id{B} \Ax{\Rightarrow} \=\Id{a} \Ax{*} \Id{b} \Id{isIn} \=\Id{A} \Id{x} \Id{B}\\
% \> \Ax{\forall} \=\Id{A}, \Id{B}, \Id{D} \Ax{:} \Id{Generic} \\
% \> \Ax{\bullet} \=(\=\Ax{\forall} \Id{a}, \Id{b} \Ax{:} \Id{RingElt} \Ax{\bullet} \=\Id{a} \Id{isIn} \Id{A} \Ax{\wedge} \=\Id{b} \Id{isIn} \Id{B} \Ax{\Rightarrow} \=\Id{a} \Ax{*} \Id{b} \Id{isIn} \Id{D}) \\
% \>\> \Ax{\Rightarrow} \Id{gcont}(\=\Id{A} \Id{x} \Id{B}, \Id{D})\\
\> \Ax{\forall} \=\Id{x}, \Id{y} \Ax{:} \Id{Generic} \Ax{\bullet} \=\Id{gcont}(\=\Id{x}, \Id{y}) \Ax{\Leftrightarrow} \=\Ax{\exists} \Id{c} \Ax{:} \Id{Generic} \Ax{\bullet} \=\Id{x} \Ax{=} \=\Id{y} \Id{x} \Id{c}\\
\> \Ax{\forall} \=\Id{p} \Ax{:} \Id{Generic} \Ax{\bullet} \=\Id{p} \Ax{\in} \Id{SimplePrime} \Ax{\Leftrightarrow} \Id{IsPrime}(\Id{p})\\
\> \Ax{\forall} \=\Id{p} \Ax{:} \Id{Generic} \\
\> \Ax{\bullet} \=\Id{IsPrime}(\Id{p}) \\
\>\> \Ax{\Leftrightarrow} \=(\=\Ax{\forall} \Id{a}, \Id{b} \Ax{:} \Id{Generic} \\
\>\>\>\> \Ax{\bullet} \=\Id{gcont}(\=\Id{a} \Id{x} \Id{b}, \Id{p}) \Ax{\Rightarrow} \=\Id{gcont}(\=\Id{a}, \Id{p}) \Ax{\vee} \Id{gcont}(\=\Id{b}, \Id{p})) \\
\>\>\> \Ax{\wedge} \Ax{\neg} \=\Id{p} \Ax{=} \Id{S}\\
\KW{end}
\end{hetcasl}

    \end{footnotesize}
  \end{mdframed}
    \caption{Colimit for prime ideals over CDR-s}
    \label{listing:prime_colim}
  \end{listing}
%\end{footnotesize}
%  \caption{Colimit Theory}
%  \label{fig:ideal_colim}
% \end{figure}
% \begin{verbatim}
% logic CASL.SulFOL=

% sorts Generic, RingElt, SimplePrime, SubSetOfRing
% sorts SimplePrime < Generic; Generic < SubSetOfRing
% op 0 : RingElt
% op 1 : RingElt
% op S : Generic
% op __*__ : RingElt * RingElt -> RingElt
% op __+__ : RingElt * RingElt -> RingElt
% op __x__ : Generic * Generic -> Generic
% pred IsIdeal : SubSetOfRing
% pred IsPrime : Generic
% pred __isIn__ : RingElt * SubSetOfRing
% pred gcont : Generic * Generic

% forall I : SubSetOfRing . I in Generic <=> IsIdeal(I)
% forall x : Generic . x x S = x 
% forall x : Generic . S x x = x 
% forall A, B : Generic.
% gcont(A, B) <=> forall a : RingElt . a isIn A => a isIn B
% forall x, y : RingElt . x + y = y + x 
% forall x, y, z : RingElt . (x + y) + z = x + (y + z)
% forall x : RingElt . x + 0 = x /\ 0 + x = x 
% forall x : RingElt . exists x' : RingElt . x' + x = 0
% forall x, y : RingElt . x * y = y * x 
% forall x, y, z : RingElt . (x * y) * z = x * (y * z)
% forall x : RingElt . x * 1 = x /\ 1 * x = x 
% forall x, y, z : RingElt
% . (x + y) * z = (x * z) + (y * z) 
% forall x, y, z : RingElt 
% . z * (x + y) = (z * x) + (z * y) 
% forall I : SubSetOfRing
% . IsIdeal(I)
%  <=> forall a, b, c : RingElt
% . ((a isIn I => a isIn S) /\ 0 isIn I)
% /\ (a isIn I /\ c isIn S => c * a isIn I)
% /\ (a isIn I /\ b isIn I /\ 
% c isIn S /\ b + c = 0 => a + c isIn I)                                              
% forall a : RingElt; A : Generic
% . a generates A
% <=> forall c : RingElt 
% . c isIn A => exists d : RingElt . c = a * d                                                  
% forall A, B : Generic; a, b : RingElt
% . a isIn A /\ b isIn B => a * b isIn A x B        
% forall A, B, D : Generic
% . (forall a, b : RingElt
% . a isIn A /\ b isIn B => a * b isIn D)
% => gcont(A x B, D)                             
% forall x, y : Generic
% . gcont(x, y) <=> exists c : Generic . x = y x c 
% forall p : Generic . p in SimplePrime <=> IsPrime(p)
% forall p : Generic
% . IsPrime(p)
% <=> (forall a, b : Generic
% . gcont(a x b, p) => gcont(a, p) \/ gcont(b, p))
% /\ not p = S                             

% \end{verbatim}


%%% Local Variables: 
%%% mode: latex
%%% TeX-master: "mathsICCC"
%%% End: 


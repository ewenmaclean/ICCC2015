\section{Background}
\label{sec:background}

\subsection{Blending in Mathematics}
\label{subsec:mathblend}

\subsection{Concept blending in mathematics}
Lakoff and N{\'u}{\~n}ez \citep{lakoff} are among the first to present
a cognitive account of the origin and development of mathematical
ideas,\footnote{This is lamented by Lakoff and N{\'u}{\~n}ez, who
claim that (prior to their work), ``there was still no discipline of
mathematical idea analysis from a cognitive perspective''
\citep{lakoff}.} arguing against the ``romantic'' style in which
mathematics is presented as an ever-increasing set of universal,
absolute, certain truths which exist independently of humans. They
present the thesis that human mathematics is grounded in bodily
experience of a physical world, and mathematical entities inherit
properties which objects in the world have, such as being consistent
or stable over time.  Exploring the physical world of object
collection might lead to concepts like the empty collection and rules
like ``adding a collection of $n$ objects to an empty collection
yields a collection with $n$ objects''. People then form grounding
metaphors between the physical world and an abstract mathematical
world, allowing us to project from everyday experiences onto abstract
concepts, thus leading to the concept of zero and the axiom that $n +
0 = n$. Lakoff and N{\'u}{\~n}ez posit that blending different
mathematical metaphors leads to more complex ideas. 

\subsection{Some mathematical concepts}
In this paper we draw on Lakoff and N{\'u}{\~n}ez's ideas to produce a
computational model of blending in abstract mathematics. We present
three illustrative examples: the mathematical notion of infinity,
prime ideals and Galois theory. The first of these is the idea of a
number, process etc without limits or end. The latter two are concepts
from abstract algebra, which explores groups and related objects and
their properties. A group is a set of elements with an operation which
takes in two of the elements and combines them to produce a third
element. The operation must satisfy four axioms:

{\em closure} (the third element must also be in the set), {\em
associativity} (the order doesn't matter for multiple applications of
the operator), {\em identity} (there exists an element in the set such
that applying the operator to that element and a second element
results in the second element) and {\em invertibility} (there exists
an element in the set such that applying the operator to that element
and a second element results in the identity element). An example of a
group is the integers under addition -- here we have:

{\em Closure:} if we sum any two integers then the result is another integer

{\em Associativity:} for any three integers, a, b, c, (a + b) + c = a +
(b + c)

{\em Identity:} this is 0: for any integer a, a + 0 = 0 + a = a

and {\em Invertibility:} for any given element a, the negation of a,
-a, is the inverse, since a + (-a) = 0.

These axioms are modified and extended, to describe fields and
rings. A {\em prime ideal} is a subset of a ring, which certain
properties, and Galois theory draws an important connection between
fields and groups, which can be used to simplify problems in field
theory. 

\subsection{Terminology for concept blending}
Our notion of concept blending is informed by Category theory, and
highly influenced by Goguen's work on concepts [ref]. In this paper we
use the terminology below.

{\bf Conceptual spaces} are partial and temporary representational
structures which are constructed on the fly when talking about a
particular situation, which are informed by the knowledge structures
associated with a domain. These are influenced by Boden's idea of a
concept space which is mapped, explored and transformed by
transcending mapped boundaries \cite{boden}, and form the input spaces
to our blend.

{\bf Axioms} are the criteria which delineate the conceptual
spaces. The axiomatic method has been a fundamental aspect of
mathematical research since Euclid, and various axiom changes have led
to revolutions in mathematics. For instance, relaxing the axioms
defining numbers led to negative and imaginary numbers, and rejecting
the parallel postulate opened up fascinating new areas of
non-Euclidean geometry.

%where heuristics for transformation, such as {\em consider
%the negative} and {\em drop a constraint}, are suggested.

{\bf A signature morphism}, {\bf colimit}, {\bf weakening}, {\bf
morphism}, {\bf generic space}, {\bf specifications}, {\bf running the
blend},

%% \subsection{Running Example}

%% \subsection{Image Schemas}  
%% \label{subsec:schemas}



%%% Local Variables: 
%%% mode: latex
%%% TeX-master: "mathsICCC"
%%% End: 

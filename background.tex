\section{Background}
\label{sec:background}

\subsection{Blending in Mathematics}
\label{subsec:mathblend}


Lakoff and N{\'u}{\~n}ez \citep{Lak00} are among the first to present
a cognitive account of the origin and development of mathematical
ideas,\footnote{This is lamented by Lakoff and N{\'u}{\~n}ez, who
claim that (prior to their work), ``there was still no discipline of
mathematical idea analysis from a cognitive perspective''
\citep{Lak00}.} arguing against the ``romantic'' style in which
mathematics is presented as an ever-increasing set of universal,
absolute, certain truths which exist independently of humans. They
present the thesis that human mathematics is grounded in bodily
experience of a physical world, and mathematical entities inherit
properties which objects in the world have, such as being consistent
or stable over time.  Exploring the physical world of object
collection might lead to concepts like the empty collection and rules
like ``adding a collection of $n$ objects to an empty collection
yields a collection with $n$ objects''. People then form grounding
metaphors between the physical world and an abstract mathematical
world, allowing us to project from everyday experiences onto abstract
concepts, thus leading to the concept of zero and the axiom that $n +
0 = n$. Lakoff and N{\'u}{\~n}ez posit that blending different
mathematical metaphors leads to more complex ideas. 

%% \subsection{Some mathematical concepts}
%% In this paper we draw on Lakoff and N{\'u}{\~n}ez's ideas to produce a
%% computational model of blending in abstract mathematics. We present
%% three illustrative examples: the mathematical notion of infinity,
%% prime ideals and Galois theory. The first of these is the idea of a
%% number, process etc without limits or end. The latter two are concepts
%% from abstract algebra, which explores groups and related objects and
%% their properties. 

%% A {\em prime ideal} is a subset of a ring, which certain
%% properties, and Galois theory draws an important connection between
%% fields and groups, which can be used to simplify problems in field
%% theory. 

\subsection{Terminology for concept blending}
Our notion of concept blending is informed by Category theory, and
highly influenced by Goguen's work on concepts [ref]. In this paper we
use the terminology below, and elucidate the terminology by means of a
running example -- discovering a version of the integers using
blending.

{\bf Conceptual spaces} are partial and temporary representational
structures which are constructed on the fly when talking about a
particular situation, which are informed by the knowledge structures
associated with a domain. These are influenced by Boden's idea of a
concept space which is mapped, explored and transformed by
transcending mapped boundaries \cite{boden}, and form the input spaces
to our blend. 

As an example of two conceptual spaces, consider one as a theory $\mathit{NAT}$ --
a theory of the natural numbers, and $\mathit{FUNC}$ -- a theory of a total
unary function with an inverse. We will refer back to these theories
in this exposition.

{\bf Axioms} are the criteria which delineate the conceptual
spaces. The axiomatic method has been a fundamental aspect of
mathematical research since Euclid, and various axiom changes have led
to revolutions in mathematics. For instance, relaxing the axioms
defining numbers led to negative and imaginary numbers, and rejecting
the parallel postulate opened up fascinating new areas of
non-Euclidean geometry.

In the conceptual space with theory $\mathit{NAT}$, an example of an axiom is $\neg \exists x. 0 =
s(x)$ -- that is that zero is the least element of the natural
numbers.  The conceptual space with theory $\mathit{FUNC}$ has an axiom
$\forall x.\;f(f^{-1}) x = x$. 
%where heuristics for transformation, such as {\em consider
%the negative} and {\em drop a constraint}, are suggested.

{\bf Signature morphisms} are mappings of symbols between two
conceptual spaces. For example $\mathit{NAT}$ contains a function $\lambda
x:\mathbb{N}.\;s(x)$ to denote the successor, and $\mathit{FUNC}$ contains a function
$\lambda x:\tau.\;f(x)$. A theory $G$ common to both may contain a function
$\lambda x:\sigma.\;g(x)$. When we show a mapping in this paper we write this
as
\begin{align}
s&&\leftarrow_{\phi(G,\mathit{NAT})}&&g&&\rightarrow_{\phi(G,\mathit{FUNC})}&&f\\
\mathbb{N}&&\leftarrow_{\phi(G,\mathit{NAT})}&&\sigma&&\rightarrow_{\phi(G,\mathit{FUNC})}&&\tau
\end{align}
\noindent The mapping $\phi(G,\mathit{NAT})$ is a signature morphism from
$G$ to $\mathit{NAT}$. Note that associated types are also mapped.

{\bf Input Spaces} refers to two or more conceptual spaces of
interest. 

{\bf Generic spaces} are conceptual spaces that possess commonality
between input spaces. 

{\bf Colimits} are conceptual spaces representing a blend of input
spaces with respect to a given generic space and set of signature
morphisms. These are uniquely computed given a generic space and set of
morphisms. There is a pictorial representation of such a computation
in our example usig theories $\mathit{NAT}$ and $\mathit{FUNC}:$
\begin{center}
  \begin{diagram}[size=7mm]
    &       &   $G$   &       & \\
    & \ldTo^{\rotatebox{-45}{$\phi(G,\mathit{NAT})$}} &       &
    \rdTo^{\rotatebox{45}{$\phi(G,\mathit{FUNC})$}} &          \\
    \SIdIndex{$\mathit{NAT}$} &       &   &       & \SIdIndex{$\mathit{FUNC}$} \\
    & \rdTo_{\rotatebox{45}{$\phi(B,\mathit{NAT})$}} &       &
    \ldTo_{\rotatebox{-45}{$\phi(B,\mathit{FUNC})$}} &  \\
    & & $Colimit$ & & 
  \end{diagram}
\end{center}
The conceptual space represented by the Colimit is often referred to
as the {\em blend}. 

{\bf Inconsistent} mathematical theories are a way of determining the
validity of a conceptual space. This is a way of evaluating whether a
blend is not only creative, but also valid. In the example of theories
$\mathit{NAT}$ and $\mathit{FUNC}$,
the computed blend is inconsistent due to the emergent axioms in the
computed colimit. The only type existing within the colimit is now
referred to as $\mathbb{Z}$ to distinguish it from the natural numbers:
\begin{eqnarray*}
\neg \exists x: \mathbb{Z}. 0 &=&s(x)\\
\forall x:\mathbb{Z}.\;s(s^{-1}) x &=& x
\end{eqnarray*}
This is an inconsistency as the second axiom states there must be an
element for which 0 is the successor.

{\bf Weakening} refers to the process of weakening the input
theories by removing symbols or axioms. If we remove the axiom 
$$
\neg \exists x: \mathbb{N}. 0 = s(x)
$$
then the resulting computed colimit contains a mathematical theory
which is consistent.

{\bf Running the blend} refers to elaborating or completing a
mathematical theory. Sometimes there are missing definitions which
need to be discovered. For example in the new theory the following
axioms exists
\begin{eqnarray*}
\forall x,y:\mathbb{Z}. s(x) + y &=& s(x+y)
\end{eqnarray*}
but we also need the axiom
\begin{eqnarray*}
\forall x,y:\mathbb{Z}. s^{-1}(x) + y &=& s^{-1}(x+y)
\end{eqnarray*}
Finding this axioms is an example of running the blend. From which it
is possible to discover and prove such theorems as
\begin{eqnarray*}
\forall x,y:\mathbb{Z}. s^{-1}(x) + s(y) &=& x+y
\end{eqnarray*}

\subsection{Technologies}

%%% FELIX??? COLIMITS, HETS, HDTP ETC...

%% \subsection{Running Example}

%% \subsection{Image Schemas}  
%% \label{subsec:schemas}



%%% Local Variables: 
%%% mode: latex
%%% TeX-master: "mathsICCC"
%%% End: 

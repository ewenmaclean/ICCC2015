\section{Blending and the infinite}
\label{sec:infinity}

%\note{Marco, Ewen, Alan, Felix}


%In this paper we want to follow Goguen's proposal \cite{Go99c} for
%implementing blending, but slightly simplified (like in
%\cite{KuMoHoBhBa12}): we will only deal with
%the well-known notion of colimit in a category \cite{Gog91}, instead of
%using $\frac{3}{2}$-colimits.
%
%In the examples here considered we will be using colimits of V-shaped
%diagrams (cf.~Figure??). Thus, in each of the cases we have to specify
%two morphisms between a source conceptual space and a target conceptual
%space, with the restriction that both morphisms share the same source
%conceptual space.
%
%Additionally we will assume that the involved conceptual spaces are
%always given using a
%CASL specification \cite{MoHaSaTa08} and that the morphisms are theorem
%preserving (but perhaps not axiom preserving). The reason to these
%assumptions
%is that in these cases it is well-known how to compute colimits
%\cite{Mo98a}; the colimit specification essentially corresponds to the
%disjoint union of the two target conceptual spaces except for not
%repeating the symbols given in the common source conceptual space.
%Moreover, we will be using 
%the HETS system \parencite{MossakowskiEA06} to compute such colimits.



%\subsection{Naturals and Integers}

\subsection{Example Revisited -- the Integers}

As a first demonstration of the machinery involved in blending
mathematical theories, we consider combining a theory of natural
numbers with the concept of the inverse of a function to obtain the
integers. Let us assume a simple partial axiomatisation of the natural
numbers (without order axioms) as shown in Listing \ref{fig:nats}, and
call this theory $\SIdIndex{Nat}$. Now let us also define a simple theory
which introduces the concept of a function with an inverse as shown in
Listing \ref{fig:inv}, and call this theory $\mathbb{F}$. 
\begin{listing}[!ht]
\begin{mdframed}
\begin{hetcasl}
\SPEC \=\SIdIndex{Nat} \Ax{=}\\
\> \SORT \Id{Nat}\\
\> \OPS \=\Id{zero} \Ax{:} \Id{Nat};\\
\>\> \Id{s} \Ax{:} \=\Id{Nat} \Ax{\rightarrow} \Id{Nat};\\
\>\> \Ax{\_\_}\Ax{+}\Ax{\_\_} \Ax{:} \=\Id{Nat} \Ax{\times} \Id{Nat} \Ax{\rightarrow} \Id{Nat}\\
\> \Ax{\forall} \=\Id{x}, \Id{y}, \Id{z} \Ax{:} \Id{Nat} \\
%\> \Ax{\bullet} \=\Id{s}(\Id{x}) \Ax{=} \Id{y} \Ax{\wedge} \=\Id{s}(\Id{x}) \Ax{=} \Id{z} \Ax{\Rightarrow} \=\Id{y} \Ax{=} \Id{z}\\
\> \Ax{\bullet} \=\Id{s}(\Id{x}) \Ax{=} \Id{s}(\Id{y}) \Ax{\Rightarrow} \=\Id{x} \Ax{=} \Id{y}\\
%\> \Ax{\bullet} \=\Ax{\exists} \Id{a} \Ax{:} \Id{Nat} \Ax{\bullet} \=\Id{s}(\Id{x}) \Ax{=} \Id{a}\\
\> \Ax{\bullet} \Ax{\neg} \=\Id{s}(\Id{x}) \Ax{=} \Id{zero}\\
\> \Ax{\bullet} \=\Id{s}(\Id{x}) \Ax{+} \Id{y} \Ax{=} \Id{s}(\=\Id{x} \Ax{+} \Id{y})\\
\> \Ax{\bullet} \=\Id{zero} \Ax{+} \Id{y} \Ax{=} \Id{y}\\
\KW{end}
\end{hetcasl}
\end{mdframed}
\caption{A theory of the natural numbers without order}
\label{fig:nats}
\end{listing}

\begin{listing}[!ht]
\begin{mdframed}
\begin{hetcasl}
\SPEC \=\SIdIndex{Func} \Ax{=}\\
\> \SORT \Id{X}\\
\> \OP \=\Id{f} \Ax{:} \=\Id{X} \Ax{\rightarrow} \Id{X}\\
\> \OP \=\Id{finv} \Ax{:} \=\Id{X} \Ax{\rightarrow} \Id{X}\\
\> \Ax{\forall} \=\Id{x} \Ax{:} \Id{X} \\
\> \Ax{\bullet} \=\Id{f}(\Id{finv}(\Id{x})) \Ax{=} \Id{x}\\
\> \Ax{\bullet} \=\Id{finv}(\Id{f}(\Id{x})) \Ax{=} \Id{x}\\
\KW{end}
\end{hetcasl}
\end{mdframed}
\caption{A theory with a function and its inverse defined}
\label{fig:inv}
\end{listing}



\subsubsection{Identifying a Generic Space}
In order to incorporate the notion of blending here we want to be able
to identify a ``generic'' component of each theory and compute the
colimit (also known as pushout).
%as discusses in \S\ref{sec:background}. 
We can use the
HDTP system \parencite{GustKS2006,Schmidt2010} to discover a common theory
and signature morphism between symbols in the
two theories $\SIdIndex{Nat}$ and $\SIdIndex{Fun}$. The Generic theory
contains a sort $N$ and a function $func$, and the morphisms from the
Generic theory to $\SIdIndex{N}$ and $\SIdIndex{Fun}$ are:
\begin{align}
%s&&\leftarrow_{g_\SIdIndex{Nat}}&&func&&\rightarrow_{g_\SIdIndex{Func}}&&f\\
%Nat&&\leftarrow_{g_\SIdIndex{Nat}}&&N&&\rightarrow_{g_\SIdIndex{Func}}&&X
s&&\leftarrow_{\phi(G,\SIdIndex{NAT})}&&func&&\rightarrow_{\phi(G,\SIdIndex{FUNC})}&&f\\
Nat&&\leftarrow_{\phi(G,\SIdIndex{NAT})}&&N&&\rightarrow_{\phi(G,\SIdIndex{FUNC})}&&X
\end{align}
Here the successor function is identified in the mapping with the
function in the theory $\SIdIndex{Fun}$, and $g_K$ (where $K$ is either
$\SIdIndex{NAT}$ or $\SIdIndex{FUNC}$) is the label for
the set of symbol mappings determined by the signature morphism from
the Generic space the theory $K$.

\subsubsection{Computing the Colimit}
The HETS system \parencite{MossakowskiEA06} can then be exploited to
find a new theory by computing the colimit:
%\begin{center}
%  \begin{diagram}[size=7mm]
%    &       &   $Gen$   &       & \\
%    & \ldTo^{\rotatebox{-45}{$g_\SIdIndex{Nat}$}} &       &
%    \rdTo^{\rotatebox{45}{$g_\SIdIndex{Fun}$}} &          \\
%    \SIdIndex{Nat} &       &   &       & \SIdIndex{Fun} \\
%    & \rdTo_{\rotatebox{45}{$b_\SIdIndex{Nat}$}} &       &
%    \ldTo_{\rotatebox{-45}{$b_\SIdIndex{Fun}$}} &  \\
%    & & $Blend$ & &
%  \end{diagram}
%\end{center}
\begin{center}
  \begin{tikzcd}[column sep=normal, row sep=small]
    & \SIdIndex{Blend}
    \\
    \SIdIndex{NAT} \arrow{ur}{b_\SIdIndex{NAT}} & & \SIdIndex{FUNC}
    \arrow{ul}[swap]{b_\SIdIndex{FUNC}} \\
    & \SIdIndex{Gen} \arrow{ul}{g_\SIdIndex{NAT}}
    \arrow{ur}[swap]{g_\SIdIndex{FUNC}}
  \end{tikzcd}
\end{center}
This generates the theory shown in Listing \ref{fig:inconsistentintegers}.

\begin{listing}[!ht]
\begin{mdframed}
%\begin{hetcasl}
%\SPEC \=\SIdIndex{Spec} \Ax{=}\\
%\> \SORT \Id{N}\\
%\> \OP \=\Ax{\_\_}\Ax{+}\Ax{\_\_} \Ax{:} \=\Id{N} \Ax{\times} \Id{N} \Ax{\rightarrow} \Id{N}\\
%\> \OP \=\Id{p} \Ax{:} \=\Id{N} \Ax{\rightarrow} \Id{N}\\
%\> \OP \=\Id{s} \Ax{:} \=\Id{N} \Ax{\rightarrow} \Id{N}\\
%\> \OP \=\Id{zero} \Ax{:} \Id{N}\\
%\> \Ax{\forall} \=\Id{x}, \Id{y}, \Id{z} \Ax{:} \Id{N} \=\Ax{\bullet} \=\Id{s}(\Id{x}) \Ax{=} \Id{y} \Ax{\wedge} \=\Id{s}(\Id{x}) \Ax{=} \Id{z} \Ax{\Rightarrow} \=\Id{y} \Ax{=} \Id{z} \`{\small{}\KW{\%}(1)\KW{\%}}\\
%\> \Ax{\forall} \=\Id{x}, \Id{y} \Ax{:} \Id{N} \=\Ax{\bullet} \=\Id{s}(\Id{x}) \Ax{=} \Id{s}(\Id{y}) \Ax{\Rightarrow} \=\Id{x} \Ax{=} \Id{y} \`{\small{}\KW{\%}(2)\KW{\%}}\\
%\> \Ax{\forall} \=\Id{x} \Ax{:} \Id{N} \=\Ax{\bullet} \=\Ax{\exists} \Id{a} \Ax{:} \Id{N} \Ax{\bullet} \=\Id{s}(\Id{x}) \Ax{=} \Id{a} \`{\small{}\KW{\%}(3)\KW{\%}}\\
%\> \Ax{\forall} \=\Id{x} \Ax{:} \Id{N} \=\Ax{\bullet} \Ax{\neg} \=\Id{s}(\Id{x}) \Ax{=} \Id{zero} \`{\small{}\KW{\%}(4)\KW{\%}}\\
%\> \Ax{\forall} \=\Id{x}, \Id{y} \Ax{:} \Id{N} \=\Ax{\bullet} \=\Id{s}(\Id{x}) \Ax{+} \Id{y} \Ax{=} \Id{s}(\=\Id{x} \Ax{+} \Id{y}) \`{\small{}\KW{\%}(5)\KW{\%}}\\
%\> \Ax{\forall} \=\Id{y} \Ax{:} \Id{N} \=\Ax{\bullet} \=\Id{zero} \Ax{+} \Id{y} \Ax{=} \Id{y} \`{\small{}\KW{\%}(6)\KW{\%}}\\
%\> \Ax{\forall} \=\Id{x} \Ax{:} \Id{N} \=\Ax{\bullet} \=\Id{s}(\Id{p}(\Id{x})) \Ax{=} \Id{x} \`{\small{}\KW{\%}(1\Ax{\_}7)\KW{\%}}\\
%\> \Ax{\forall} \=\Id{x} \Ax{:} \Id{N} \=\Ax{\bullet}
%\=\Id{p}(\Id{s}(\Id{x})) \Ax{=} \Id{x}
%\`{\small{}\KW{\%}(2\Ax{\_}8)\KW{\%}}\\
%\KW{end}
%\end{hetcasl}
\begin{hetcasl}
\SPEC \=\SIdIndex{Spec} \Ax{=}\\
\> \SORT \Id{N}\\
\> \OP \=\Ax{\_\_}\Ax{+}\Ax{\_\_} \Ax{:} \=\Id{N} \Ax{\times} \Id{N} \Ax{\rightarrow} \Id{N}\\
\> \OP \=\Id{p} \Ax{:} \=\Id{N} \Ax{\rightarrow} \Id{N}\\
\> \OP \=\Id{s} \Ax{:} \=\Id{N} \Ax{\rightarrow} \Id{N}\\
\> \OP \=\Id{zero} \Ax{:} \Id{N}\\
%\> \Ax{\forall} \=\Id{x}, \Id{y}, \Id{z} \Ax{:} \Id{N} \=\Ax{\bullet} \=\Id{s}(\Id{x}) \Ax{=} \Id{y} \Ax{\wedge} \=\Id{s}(\Id{x}) \Ax{=} \Id{z} \Ax{\Rightarrow} \=\Id{y} \Ax{=} \Id{z} \`{\small{}\KW{\%}(1)\KW{\%}}\\
\> \Ax{\forall} \=\Id{x}, \Id{y} \Ax{:} \Id{N} \=\Ax{\bullet} \=\Id{s}(\Id{x}) \Ax{=} \Id{s}(\Id{y}) \Ax{\Rightarrow} \=\Id{x} \Ax{=} \Id{y} \`{\small{}\KW{\%}(1)\KW{\%}}\\
%\> \Ax{\forall} \=\Id{x} \Ax{:} \Id{N} \=\Ax{\bullet} \=\Ax{\exists} \Id{a} \Ax{:} \Id{N} \Ax{\bullet} \=\Id{s}(\Id{x}) \Ax{=} \Id{a} \`{\small{}\KW{\%}(3)\KW{\%}}\\
\> \Ax{\forall} \=\Id{x} \Ax{:} \Id{N} \=\Ax{\bullet} \Ax{\neg} \=\Id{s}(\Id{x}) \Ax{=} \Id{zero} \`{\small{}\KW{\%}(2)\KW{\%}}\\
\> \Ax{\forall} \=\Id{x}, \Id{y} \Ax{:} \Id{N} \=\Ax{\bullet} \=\Id{s}(\Id{x}) \Ax{+} \Id{y} \Ax{=} \Id{s}(\=\Id{x} \Ax{+} \Id{y}) \`{\small{}\KW{\%}(3)\KW{\%}}\\
\> \Ax{\forall} \=\Id{y} \Ax{:} \Id{N} \=\Ax{\bullet} \=\Id{zero} \Ax{+} \Id{y} \Ax{=} \Id{y} \`{\small{}\KW{\%}(4)\KW{\%}}\\
\> \Ax{\forall} \=\Id{x} \Ax{:} \Id{N} \=\Ax{\bullet} \=\Id{s}(\Id{p}(\Id{x})) \Ax{=} \Id{x} \`{\small{}\KW{\%}(1\Ax{\_}5)\KW{\%}}\\
\> \Ax{\forall} \=\Id{x} \Ax{:} \Id{N} \=\Ax{\bullet}
\=\Id{p}(\Id{s}(\Id{x})) \Ax{=} \Id{x}
\`{\small{}\KW{\%}(2\Ax{\_}6)\KW{\%}}\\
\KW{end}
\end{hetcasl}
\end{mdframed}
\caption{An inconsistent partial approach to the integers (without order)}
\label{fig:inconsistentintegers}
\end{listing}


\subsubsection{Removal of Inconsistencies}
This theory is automatically determined to be inconsistent due to the axioms
\begin{eqnarray}
\forall x : \mathbb{Z} . \neg s(x) &=& 0 \label{eq:limnat1}\\
s(p(x)) &=& x \label{eq:sucpre}
\end{eqnarray}
Removal of the limiting axiom (\ref{eq:limnat1}) results in a theory which is very
similar to what we understand to be the integers as shown in Listing~\ref{fig:integers}.
\begin{listing}[!ht]
\begin{mdframed}
%\begin{hetcasl}
%\SPEC \=\SIdIndex{Spec} \Ax{=}\\
%\> \SORT \Id{N}\\
%\> \OP \=\Ax{\_\_}\Ax{+}\Ax{\_\_} \Ax{:} \=\Id{N} \Ax{\times} \Id{N} \Ax{\rightarrow} \Id{N}\\
%\> \OP \=\Id{p} \Ax{:} \=\Id{N} \Ax{\rightarrow} \Id{N}\\
%\> \OP \=\Id{s} \Ax{:} \=\Id{N} \Ax{\rightarrow} \Id{N}\\
%\> \OP \=\Id{zero} \Ax{:} \Id{N}\\
%\> \Ax{\forall} \=\Id{x}, \Id{y}, \Id{z} \Ax{:} \Id{N} \=\Ax{\bullet} \=\Id{s}(\Id{x}) \Ax{=} \Id{y} \Ax{\wedge} \=\Id{s}(\Id{x}) \Ax{=} \Id{z} \Ax{\Rightarrow} \=\Id{y} \Ax{=} \Id{z} \`{\small{}\KW{\%}(1)\KW{\%}}\\
%\> \Ax{\forall} \=\Id{x}, \Id{y} \Ax{:} \Id{N} \=\Ax{\bullet} \=\Id{s}(\Id{x}) \Ax{=} \Id{s}(\Id{y}) \Ax{\Rightarrow} \=\Id{x} \Ax{=} \Id{y} \`{\small{}\KW{\%}(2)\KW{\%}}\\
%\> \Ax{\forall} \=\Id{x} \Ax{:} \Id{N} \=\Ax{\bullet} \=\Ax{\exists} \Id{a} \Ax{:} \Id{N} \Ax{\bullet} \=\Id{s}(\Id{x}) \Ax{=} \Id{a} \`{\small{}\KW{\%}(3)\KW{\%}}\\
%\> \Ax{\forall} \=\Id{x}, \Id{y} \Ax{:} \Id{N} \=\Ax{\bullet} \=\Id{s}(\Id{x}) \Ax{+} \Id{y} \Ax{=} \Id{s}(\=\Id{x} \Ax{+} \Id{y}) \`{\small{}\KW{\%}(4)\KW{\%}}\\
%\> \Ax{\forall} \=\Id{y} \Ax{:} \Id{N} \=\Ax{\bullet} \=\Id{zero} \Ax{+} \Id{y} \Ax{=} \Id{y} \`{\small{}\KW{\%}(5)\KW{\%}}\\
%\> \Ax{\forall} \=\Id{x} \Ax{:} \Id{N} \=\Ax{\bullet} \=\Id{s}(\Id{p}(\Id{x})) \Ax{=} \Id{x} \`{\small{}\KW{\%}(1\Ax{\_}7)\KW{\%}}\\
%\> \Ax{\forall} \=\Id{x} \Ax{:} \Id{N} \=\Ax{\bullet} \=\Id{p}(\Id{s}(\Id{x})) \Ax{=} \Id{x} \`{\small{}\KW{\%}(2\Ax{\_}8)\KW{\%}}\\
%\KW{end}
%\end{hetcasl}
\begin{hetcasl}
\SPEC \=\SIdIndex{Spec} \Ax{=}\\
\> \SORT \Id{N}\\
\> \OP \=\Ax{\_\_}\Ax{+}\Ax{\_\_} \Ax{:} \=\Id{N} \Ax{\times} \Id{N} \Ax{\rightarrow} \Id{N}\\
\> \OP \=\Id{p} \Ax{:} \=\Id{N} \Ax{\rightarrow} \Id{N}\\
\> \OP \=\Id{s} \Ax{:} \=\Id{N} \Ax{\rightarrow} \Id{N}\\
\> \OP \=\Id{zero} \Ax{:} \Id{N}\\
%\> \Ax{\forall} \=\Id{x}, \Id{y}, \Id{z} \Ax{:} \Id{N} \=\Ax{\bullet} \=\Id{s}(\Id{x}) \Ax{=} \Id{y} \Ax{\wedge} \=\Id{s}(\Id{x}) \Ax{=} \Id{z} \Ax{\Rightarrow} \=\Id{y} \Ax{=} \Id{z} \`{\small{}\KW{\%}(1)\KW{\%}}\\
\> \Ax{\forall} \=\Id{x}, \Id{y} \Ax{:} \Id{N} \=\Ax{\bullet} \=\Id{s}(\Id{x}) \Ax{=} \Id{s}(\Id{y}) \Ax{\Rightarrow} \=\Id{x} \Ax{=} \Id{y} \`{\small{}\KW{\%}(1)\KW{\%}}\\
%\> \Ax{\forall} \=\Id{x} \Ax{:} \Id{N} \=\Ax{\bullet} \=\Ax{\exists} \Id{a} \Ax{:} \Id{N} \Ax{\bullet} \=\Id{s}(\Id{x}) \Ax{=} \Id{a} \`{\small{}\KW{\%}(3)\KW{\%}}\\
\> \Ax{\forall} \=\Id{x}, \Id{y} \Ax{:} \Id{N} \=\Ax{\bullet} \=\Id{s}(\Id{x}) \Ax{+} \Id{y} \Ax{=} \Id{s}(\=\Id{x} \Ax{+} \Id{y}) \`{\small{}\KW{\%}(2)\KW{\%}}\\
\> \Ax{\forall} \=\Id{y} \Ax{:} \Id{N} \=\Ax{\bullet} \=\Id{zero} \Ax{+} \Id{y} \Ax{=} \Id{y} \`{\small{}\KW{\%}(3)\KW{\%}}\\
\> \Ax{\forall} \=\Id{x} \Ax{:} \Id{N} \=\Ax{\bullet} \=\Id{s}(\Id{p}(\Id{x})) \Ax{=} \Id{x} \`{\small{}\KW{\%}(1\Ax{\_}4)\KW{\%}}\\
\> \Ax{\forall} \=\Id{x} \Ax{:} \Id{N} \=\Ax{\bullet} \=\Id{p}(\Id{s}(\Id{x})) \Ax{=} \Id{x} \`{\small{}\KW{\%}(2\Ax{\_}5)\KW{\%}}\\
\KW{end}
\end{hetcasl}
\end{mdframed}
\caption{A consistent partial approach to the integers (without order)}
\label{fig:integers}
\end{listing}

\subsubsection{Running the Blend}

Running the blend refers to discovering axioms or definitions which
make the blend incomplete. In the example of the version in Listing
\ref{fig:integers}, the definition of plus needs to be extended to
understand how to calculate with the predecessor function:
$$
p(x) + y = p(x+y)
$$
\noindent from which theorems such as 
$$
p(x) + s(y) = x+y
$$
\noindent can be proved.



%%% Local Variables: 
%%% mode: latex
%%% TeX-master: "mathsICCC"
%%% End: 

\section{Blending and the infinite}
\label{sec:infinity}

\note{Marco, Ewen, Alan, Felix}

\subsection{Naturals and Integers}

\subsection{A Simple Example -- the Integers}

In order to demonstrate the machinery involved in blending
mathematical theories, we consider combining a theory of natural
numbers with the concept of the inverse of a function to obtain the
integers. Let us assume an simple axiomatisation of the natural
numbers (without order axioms) as shown in Figure \ref{fig:nats}, and
call this theory $\mathbb{N}$. Now let us also define a simple theory
which introduces the concept of a function with an inverse as shown in
Figure \ref{fig:inv}, and call this theory $\mathbb{F}$. 
\begin{figure}[!ht]
\begin{hetcasl}
\SPEC \=\SIdIndex{Nat} \Ax{=}\\
\> \SORT \Id{Nat}\\
\> \OPS \=\Id{zero} \Ax{:} \Id{Nat};\\
\>\> \Id{s} \Ax{:} \=\Id{Nat} \Ax{\rightarrow} \Id{Nat};\\
\>\> \Ax{\_\_}\Ax{+}\Ax{\_\_} \Ax{:} \=\Id{Nat} \Ax{\times} \Id{Nat} \Ax{\rightarrow} \Id{Nat}\\
\> \Ax{\forall} \=\Id{x}, \Id{y}, \Id{z} \Ax{:} \Id{Nat} \\
\> \Ax{\bullet} \=\Id{s}(\Id{x}) \Ax{=} \Id{y} \Ax{\wedge} \=\Id{s}(\Id{x}) \Ax{=} \Id{z} \Ax{\Rightarrow} \=\Id{y} \Ax{=} \Id{z}\\
\> \Ax{\bullet} \=\Id{s}(\Id{x}) \Ax{=} \Id{s}(\Id{y}) \Ax{\Rightarrow} \=\Id{x} \Ax{=} \Id{y}\\
\> \Ax{\bullet} \=\Ax{\exists} \Id{a} \Ax{:} \Id{Nat} \Ax{\bullet} \=\Id{s}(\Id{x}) \Ax{=} \Id{a}\\
\> \Ax{\bullet} \Ax{\neg} \=\Id{s}(\Id{x}) \Ax{=} \Id{zero}\\
\> \Ax{\bullet} \=\Id{s}(\Id{x}) \Ax{+} \Id{y} \Ax{=} \Id{s}(\=\Id{x} \Ax{+} \Id{y})\\
\> \Ax{\bullet} \=\Id{zero} \Ax{+} \Id{y} \Ax{=} \Id{y}\\
\KW{end}
\end{hetcasl}
\caption{A theory of the natural numbers without order}
\label{fig:nats}
\end{figure}

\begin{figure}[!ht]
\begin{hetcasl}
\SPEC \=\SIdIndex{Func} \Ax{=}\\
\> \SORT \Id{X}\\
\> \OP \=\Id{f} \Ax{:} \=\Id{X} \Ax{\rightarrow} \Id{X}\\
\> \OP \=\Id{finv} \Ax{:} \=\Id{X} \Ax{\rightarrow} \Id{X}\\
\> \Ax{\forall} \=\Id{x} \Ax{:} \Id{X} \\
\> \Ax{\bullet} \=\Id{f}(\Id{finv}(\Id{x})) \Ax{=} \Id{x}\\
\> \Ax{\bullet} \=\Id{finv}(\Id{f}(\Id{x})) \Ax{=} \Id{x}\\
\KW{end}
\end{hetcasl}
\caption{A theory with a function and its inverse defined}
\label{fig:inv}
\end{figure}



\subsubsection{Identifying a Generic Space}
In order to incorporate the notion of blending here we want to be able
to identify a ``generic'' component of each theory and compute the
pushout as discusses in \S\ref{sec:blendingtheory}. We can use the
HDTP system \cite{GustKS2006,Schmidt2010} to discover a common theory
and signature morphism between symbols in the
two theories $\mathbb{N}$ and $\mathbb{F}$. The Generic theory
contains a sort $N$ and a function $func$, and the morphisms from the
Generic theory to $\mathbb{N}$ and $\mathbb{F}$ are:
\begin{align}
s&&\leftarrow_{g_\mathbb{N}}&&func&&\rightarrow_{g_\mathbb{F}}&&f\\
Nat&&\leftarrow_{g_\mathbb{N}}&&N&&\rightarrow_{g_\mathbb{F}}&&X
\end{align}
Here the successor function is identified in the mapping with the
function in the theory $\mathbb{F}$, and $g_K$ is the label for
the set of symbol mappings determined by the signature morphism from
the Generic space the theory $K$.

\subsubsection{Computing the Colimit}
The HETS system\cite{MossakowskiEA06} can then be exploited to find a new theory by computing the colimit:
\begin{center}
  \begin{diagram}[size=7mm]
    &       &   $Gen$   &       & \\
    & \ldTo^{\rotatebox{-45}{$g_\mathbb{N}$}} &       & \rdTo^{\rotatebox{45}{$g_\mathbb{F}$}} &          \\
    \mathbb{N} &       &   &       & \mathbb{F} \\
    & \rdTo_{\rotatebox{45}{$b_\mathbb{N}$}} &       & \ldTo_{\rotatebox{-45}{$b_\mathbb{F}$}} &  \\
    & & $Blend$ & &
  \end{diagram}
\end{center}
This generates the theory shown in \ref{fig:inconsistentintegers}.

\begin{figure}[!ht]
\begin{hetcasl}
\SPEC \=\SIdIndex{Spec} \Ax{=}\\
\> \SORT \Id{N}\\
\> \OP \=\Ax{\_\_}\Ax{+}\Ax{\_\_} \Ax{:} \=\Id{N} \Ax{\times} \Id{N} \Ax{\rightarrow} \Id{N}\\
\> \OP \=\Id{p} \Ax{:} \=\Id{N} \Ax{\rightarrow} \Id{N}\\
\> \OP \=\Id{s} \Ax{:} \=\Id{N} \Ax{\rightarrow} \Id{N}\\
\> \OP \=\Id{zero} \Ax{:} \Id{N}\\
\> \Ax{\forall} \=\Id{x}, \Id{y}, \Id{z} \Ax{:} \Id{N} \=\Ax{\bullet} \=\Id{s}(\Id{x}) \Ax{=} \Id{y} \Ax{\wedge} \=\Id{s}(\Id{x}) \Ax{=} \Id{z} \Ax{\Rightarrow} \=\Id{y} \Ax{=} \Id{z} \`{\small{}\KW{\%}(Ax1)\KW{\%}}\\
\> \Ax{\forall} \=\Id{x}, \Id{y} \Ax{:} \Id{N} \=\Ax{\bullet} \=\Id{s}(\Id{x}) \Ax{=} \Id{s}(\Id{y}) \Ax{\Rightarrow} \=\Id{x} \Ax{=} \Id{y} \`{\small{}\KW{\%}(Ax2)\KW{\%}}\\
\> \Ax{\forall} \=\Id{x} \Ax{:} \Id{N} \=\Ax{\bullet} \=\Ax{\exists} \Id{a} \Ax{:} \Id{N} \Ax{\bullet} \=\Id{s}(\Id{x}) \Ax{=} \Id{a} \`{\small{}\KW{\%}(Ax3)\KW{\%}}\\
\> \Ax{\forall} \=\Id{x} \Ax{:} \Id{N} \=\Ax{\bullet} \Ax{\neg} \=\Id{s}(\Id{x}) \Ax{=} \Id{zero} \`{\small{}\KW{\%}(Ax4)\KW{\%}}\\
\> \Ax{\forall} \=\Id{x}, \Id{y} \Ax{:} \Id{N} \=\Ax{\bullet} \=\Id{s}(\Id{x}) \Ax{+} \Id{y} \Ax{=} \Id{s}(\=\Id{x} \Ax{+} \Id{y}) \`{\small{}\KW{\%}(Ax5)\KW{\%}}\\
\> \Ax{\forall} \=\Id{y} \Ax{:} \Id{N} \=\Ax{\bullet} \=\Id{zero} \Ax{+} \Id{y} \Ax{=} \Id{y} \`{\small{}\KW{\%}(Ax6)\KW{\%}}\\
\> \Ax{\forall} \=\Id{x} \Ax{:} \Id{N} \=\Ax{\bullet} \=\Id{s}(\Id{p}(\Id{x})) \Ax{=} \Id{x} \`{\small{}\KW{\%}(Ax1\Ax{\_}7)\KW{\%}}\\
\> \Ax{\forall} \=\Id{x} \Ax{:} \Id{N} \=\Ax{\bullet}
\=\Id{p}(\Id{s}(\Id{x})) \Ax{=} \Id{x}
\`{\small{}\KW{\%}(Ax2\Ax{\_}8)\KW{\%}}
\end{hetcasl}
\caption{An inconsistent version of the integers (without order)}
\label{fig:inconsistentintegers}
\end{figure}




\subsubsection{Removal of Inconsistencies}
This theory is automatically determined to be inconsistent due to the axioms
\begin{eqnarray}
\forall x : \mathbb{Z} . not s(x) &=& 0 \label{eq:limnat1}\\
s(p(x)) &=& x \label{eq:sucpre}
\end{eqnarray}
removal of the limiting axiom (\ref{eq:limnat1}) results in a theory which is very
similar to what we understand to be the integers as shown in Figure \ref{fig:integers}.
\begin{figure}[!ht]
\begin{hetcasl}
\SPEC \=\SIdIndex{Spec} \Ax{=}\\
\> \SORT \Id{N}\\
\> \OP \=\Ax{\_\_}\Ax{+}\Ax{\_\_} \Ax{:} \=\Id{N} \Ax{\times} \Id{N} \Ax{\rightarrow} \Id{N}\\
\> \OP \=\Id{p} \Ax{:} \=\Id{N} \Ax{\rightarrow} \Id{N}\\
\> \OP \=\Id{s} \Ax{:} \=\Id{N} \Ax{\rightarrow} \Id{N}\\
\> \OP \=\Id{zero} \Ax{:} \Id{N}\\
\> \Ax{\forall} \=\Id{x}, \Id{y}, \Id{z} \Ax{:} \Id{N} \=\Ax{\bullet} \=\Id{s}(\Id{x}) \Ax{=} \Id{y} \Ax{\wedge} \=\Id{s}(\Id{x}) \Ax{=} \Id{z} \Ax{\Rightarrow} \=\Id{y} \Ax{=} \Id{z} \`{\small{}\KW{\%}(Ax1)\KW{\%}}\\
\> \Ax{\forall} \=\Id{x}, \Id{y} \Ax{:} \Id{N} \=\Ax{\bullet} \=\Id{s}(\Id{x}) \Ax{=} \Id{s}(\Id{y}) \Ax{\Rightarrow} \=\Id{x} \Ax{=} \Id{y} \`{\small{}\KW{\%}(Ax2)\KW{\%}}\\
\> \Ax{\forall} \=\Id{x} \Ax{:} \Id{N} \=\Ax{\bullet} \=\Ax{\exists} \Id{a} \Ax{:} \Id{N} \Ax{\bullet} \=\Id{s}(\Id{x}) \Ax{=} \Id{a} \`{\small{}\KW{\%}(Ax3)\KW{\%}}\\
\> \Ax{\forall} \=\Id{x}, \Id{y} \Ax{:} \Id{N} \=\Ax{\bullet} \=\Id{s}(\Id{x}) \Ax{+} \Id{y} \Ax{=} \Id{s}(\=\Id{x} \Ax{+} \Id{y}) \`{\small{}\KW{\%}(Ax4)\KW{\%}}\\
\> \Ax{\forall} \=\Id{y} \Ax{:} \Id{N} \=\Ax{\bullet} \=\Id{zero} \Ax{+} \Id{y} \Ax{=} \Id{y} \`{\small{}\KW{\%}(Ax5)\KW{\%}}\\
\> \Ax{\forall} \=\Id{x} \Ax{:} \Id{N} \=\Ax{\bullet} \=\Id{s}(\Id{p}(\Id{x})) \Ax{=} \Id{x} \`{\small{}\KW{\%}(Ax1\Ax{\_}7)\KW{\%}}\\
\> \Ax{\forall} \=\Id{x} \Ax{:} \Id{N} \=\Ax{\bullet} \=\Id{p}(\Id{s}(\Id{x})) \Ax{=} \Id{x} \`{\small{}\KW{\%}(Ax2\Ax{\_}8)\KW{\%}}\\
\KW{end}
\end{hetcasl}
\caption{A consistent version of the integers (without order)}
\label{fig:integers}
\end{figure}

\subsubsection{Running the Blend}

Running the blend refers to discovering axioms or definitions which
make the blend incomplete. In the example of the version in Figure
\ref{fig:integers}, the definition of plus needs to be extended to
understand how to calculate with the predecessor function:
$$
p(x) + y = p(x+y)
$$
\noindent from which theorems such as 
$$
p(x) + s(y) = x+y
$$
\noindent can be discovered.


\subsection{Potential and actual infinity}


%%% Local Variables: 
%%% mode: latex
%%% TeX-master: "mathsICCC"
%%% End: 

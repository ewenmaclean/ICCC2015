\subsection{Issues raised}

We have investigated the possibility of formalising Galois theory, and
note here some of the possibilties for blend computations, along with
some issues we need to address.

\subsubsection{Field Extension as a Blend}

The first possibility here is that one could exploit blending
machinery in order to discover the notion of a field extension. Take a
field $F$ and adjoining it with an element $a$ to form $F(a)$ is a
form of a blend of the conceptual space containing the field $F$ and
the conceptual space containing element $a$. Equally for a field
extension one could envisage blending a conceptual space containing
field $F$ with a conceptual space containing a set of elements
$E$. The field axioms then hold for $F$ and $E/F$, but not for the
elements of $E$.

Although this seems a natural blend, in reality when reasoning about
polynomials with rational coefficients, it is useful to distinguish
the three separate types -- those of $E$, those of $F$ and those of
$E/F$ as a supertype. Using blending machinery removes the distinction
between these types.

\subsubsection{Splitting Field Extension Theorem}

We would like to be able to use blending to discover that there exists a
field $F$ extended with roots of polynomial $f(x)$, where the coefficients of $f(x)$ are in $F$. The more challenging but creative step is to discover that extending $F$ {\em only} with the roots of $f(x)$ forms a field. We have investigated the possibility of blending the concept of the root of a polynomial with rational coefficients with a field of rationals and then trying to prove the theorem from the resulting axioms. Discovery of the theorem itself is the real creative step and is an example of running the blend. This is ambitious and is further work.

\subsubsection{Automorphisms}

In order to discover the most interesting and surprising Galois
theorem, that automorphisms fixing the field $F$ permute the elements
of $E$ in field extension $E/F$, we need to be able formalise the
notion of an automorphism. The most natural way to do this is in the
machinery that we currently use is to define an automorphism $\alpha$
as a higher order function: $\alpha:\; E/F \to E/F$. Currently there
is no way of computing colimits with higher order symbols
mechanically, so we are constrained to do this on paper.
%% There are various technical questions \textbf{(@Alan, @Ewen, @Felix)},
%% but from a naive mathematical standpoint the first issue is: is it
%% always clear how to combine the relevant facts?  And a related
%% question: is it always clear what the relevant facts actually are?

%% If blending is the realisation of ``combinatorial creativity'' why are
%% we not swamped by the combinatorial explosion of possible things to
%% combine?

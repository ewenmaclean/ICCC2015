\section{Conclusions}
\label{sec:conc}

The examples presented in this paper trace the development of the
blending approach.
%% The early examples illustrate the reconstruction
%% of certain mathematical objects.  The middle examples begin to ``kick
%% back'' -- the blend provides an unexpected axiom, which turns out to
%% be mathematically provocative.  We noticed also that ``partial
%% combinations'' are a better reflection of mathematical practice than
%% ``total combinations.''  Finally, the future-oriented challenge
%% question poses several problems for implementation, but also suggests
%% that as it develops, the blending approach can fruitfully combine
%% computer mathematics with learning science.  As regards computational
%% creativity, 
the current paper begins with reconstructions, but also
quickly shows how computed blends can suggest new mathematical
definitions and concepts of interest to practising mathematicians.
The analysis offered here shows that this work is a building block that
will be useful for future developments that are able to reason more
flexibly about mathematical problems -- and systematically find and
propose new concepts and problems.

%%% Local Variables: 
%%% mode: latex
%%% TeX-master: "mathsICCC"
%%% End: 
